
\begin{abstract}

	Mots clés : interprétation d'image, fenêtrage, moteur d'inférence, topologie, photométrie.\\

	Ce rapport fait état de l'art en matière d'interprétation d'image guidée par des connaissances topologiques et photométriques à priori. L'utilisation d'informations à priori est un moyen d'anticiper le contenu d'une image et donc d'en optimiser son traitement. On représente ces connaissances sous forme de graphes conceptuels à partir desquels on infère le contenu de l'image. Il s'agit plus concrètement du nombre de classes de l'image et de leurs relations photométriques, à tout instant d'une procédure de segmentation itérative. Un point important de notre étude est la limitation à des connaissances non quantitatives de manière à appliquer un traitement spécifique à chaque image et à prendre en compte ses particularités.

	Dans un second temps, nous quantifions les bénéfices de la méthode proposée à savoir, la diminution des données polluantes, le gain de temps de calcul ou encore sa robustesse.
	
	Une dernière partie est consacrée à l'application à des images médicales de la région de l'abdomen dans lesquelles on fait apparaître clairement les tumeurs du foie et les vaisseaux hépatiques dans le but d'assister le diagnostic médical.


\vspace{5em}

	Keywords: image interpretation, windowing, inference engine, topology, photometry.\\

	This report is the state of the art of image interpretation led by topological and photometrical informations. Using a priori informations is a method to anticipate the content of an image and then to improve the processing. We represent these informations as conceptual graphs and we infer image content. It is more precisely the number of classes in the image and their photometric relations, at each iteration of a segmentation procedure. One important point is that we consider only non quantitative informations in order to specify the image processing and consider its particularity.
	
	Then we quantify the benefits of the proposed method namely polluted data reduction, processing time improvement and strength.
	
	The last part of this work is an application to medical images from abdomen that we proceed in order to show clearly liver tumors and vessel system for diagnostic assistance purpose.
 	
\end{abstract}
